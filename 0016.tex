\subsection{围裙与手制料理是男人的浪漫}

真昼同意在周家里做饭的同时,提出了如下条件。\\

\begin{itemize}
    \item 周出材料费的半数加上若干人工费。
    \item 如果有事不能一起吃饭至少提前一天通知对方。
    \item 食材的采购和饭后的处理由两人分担。
\end{itemize}

关于第一条中的人工费,是周不好意思占用真昼的时间所以才提出的。在这一点上真昼做出了让步,而其它部分则没有发生什么争执,顺利地决定了下来。\\

至于让真昼来做饭这一点,由于早就是既定事项,所以并没有什么可烦恼的。\\

于是在这么决定好的第二天,真昼便早早地拎着——准确来说是两只手抱着购物袋来到周的家做起了下厨的准备。\\

「……还真的都新到几乎没有使用痕迹呢……」

「啰嗦」\\

家中有一位穿着围裙的女性。周明明身处于这种好似男人浪漫的具现一般的状况,却不知为何感到如坐针毡。\\

将头发扎成一束的真昼带来的新鲜感也是理由之一,但主要原因还是在于厨房基本就没使用过这点被真昼再次指出所造成的尴尬吧。\\

「明明有这么多好东西却放着吃灰」

「你能用上的话那不就不吃灰了么」

「那只是结果论。这么好的厨具都因为怀才不遇哭出来了」

「那就用你拿手的厨艺让它们破涕为笑吧」\\

周干脆地表达自己不行,真昼则一脸无语地看着他,但也许是料到如此,周只是叹了口气而并没有抱怨什么。\\

「那么,有做饭用的调味料吗」

「有啊,你当我傻吗。保存方法和保质期也都没问题」

「哎呀真是意外」

「因为都没开封呐」\\

大部分调味品都以未开封状态被放在阴凉避光的地方,所以应该不必担心吧。

明明都买来了,这些东西却得不到展现自己的机会。实际上由于周基本没下过厨房,所以压根就没动过它们。对调味料来说,能被真昼这位厨师使用,应该也算是物尽其用了。\\

「这可不是什么好自豪的事情。不过,要是不够的话我回家拿来用就好」

「帮大忙了」

「总之既然有基本的调味料,那应该多少能做出点东西。啊,今天的菜单我擅自定下来了,没关系吧」

「反正我不太清楚这些东西,能吃的话什么都行。我也不怎么挑食」

「这样啊。那我就动手了……请告诉我一下调味料放的地方」

「都放在这个篮子里」

「……还真的都没开封呢……」\\

真昼瞄了一眼塞满调味料的篮子,无语地皱了皱眉,不过因为周事先说过,她马上便恢复到原先的表情,到水龙头旁边洗起手来了。\\

「那我就开始做饭了。你就在客厅或者房间里等着就好」

「行。反正我也帮不上忙」

「还真是干脆呢……不过也好,要是你不会料理还晃悠来晃悠去的我也很难办」

「你也很直接啊」

「毕竟是事实。跟你也没有必要拐弯抹角的吧」\\

正如真昼所说,自己显然是个累赘,于是周老实地走回客厅观察起真昼的背影。\\

真昼洗完手后就迅速投入到了调理工作中。\\

虽然不知道她要做什么,但从准备好的材料看应该是日式餐点。

能在自己家让真昼做出那些美味的料理,周不禁感到有些不可思议,甚至怀疑自己是否在做梦。然而他看到真昼摇晃起背后扎成一束的秀发处理着食材,就知道了一切都是现实。\\

(……怎么说呢,感觉就跟有了老婆一样)\\

尽管两个人彼此都没有这样的感情,但眼前的状态看上去实在像自己已成了家一样,让周不由得心生联想。\\

周自然是对真昼没有一丝一毫的非分之想,不过有个美少女在自家厨房这个状况本身就足够让人浮想联翩了。\\

果然,不论是否抱有好感,可爱的少女愿意为自己做饭这一场景,都足以让周的胸口产生一丝悸动。\\

「……你不会在想些乱七八糟的事吧?」

「别瞎猜啊」\\

真昼头也不回的突然发问让周差点面部抽筋,但也幸亏真昼没有回头才让此事不至于败露。\\

这家伙还真是敏锐啊——周心生佩服、感到背脊发凉的同时,也收起了微微涌出但还尚未形成邪念的男人心,继续观察起了真昼的背影。
